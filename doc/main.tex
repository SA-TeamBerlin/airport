\documentclass{article}
\usepackage[utf8]{inputenc}

\title{User Guide}
\author{Andrian Hevalo, Myroslava Romanuik, Daryna Dzhala,\and Yaroslav Boiko, Oleksandr Potapov}
\date{December 2019}

\begin{document}

\maketitle

\section{Welcome to Airport Project!}
The file you are reading was written for you to make project easy to use and understand the idea behind. It is a quick guide into project techniques, features, user stories and how to start using it.

\section{Project Introduction}

In general, the project is an Java application made for airports.

Our Airport project (TG) is a software development technology, which has been developed by a bunch of dedicated individuals from Ukraine and Australia with a financial support of Fielden Management Services Pty. Ltd. Basically, the project itself was already implemented, but the problem was to extend its functionality be implementing user stories. 

We've built TG for our own purpose in order to remove many low level technical obstacles that are often associated with building sophisticated transnational applications such as Enterprise Asset Maintenance (EAM) or Enterprise Resource Planning (ERP) systems. And we have successfully used TG for developing large-scale and mission-critical enterprise applications.

We look at the system design as the decomposition problem, which can be solved successfully with the right approach in the right context. TG establishes a well defined pattern for modelling business domains, including domain entities, processes and their inter-dependencies (down to the level of properties with automatic re validation support). At the code level the description of the domain model is achieved by using a small set of base classes and various annotations. Together these constitute what we refer to as "Entity Definition Language" (EDL), which is complemented by "Entity Query Language" (EQL) for data querying (relational databases).


\section{Technologies used and characteristics}
The project was originally implemented in Java programming language. It uses Maven tools and should be run in Eclipse. For data storage it uses relational database and server for quick data processing. Also for installation flexibility it uses Docker and HAProxy.


\end{document}
